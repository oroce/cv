%!TEX TS-program = xelatex
\documentclass[]{friggeri-cv}
\usepackage{sourcesanspro}
\newfontfamily\bodyfont[]{SourceSansPro}
\newfontfamily\thinfont[]{SourceSansPro ExtraLight}
\newfontfamily\headingfont[]{SourceSansPro Black}

\defaultfontfeatures{Mapping=tex-text}
\setmainfont[Mapping=tex-text, Color=textcolor]{SourceSansPro Light}
\usepackage{xcolor}
\addbibresource{bibliography.bib}

\begin{document}
\header{robert}{oroszi}
       {web and mobile developer}

% In the aside, each new line forces a line break
\begin{aside}
  \section{about}
    Budapest
    Hungary
    ~
    \href{mailto:robert@oroszi.net}{robert@oroszi.net}
    \href{http://oroszi.net}{http://oroszi.net}
    \href{http://twitter.com/oroce}{twitter://oroce}
	\href{http://github.com/oroce}{github://oroce}
	\href{http://bitbucket.org/oroce}{bitbucket://oroce}
  \section{languages}
    native hungarian
	english
	german
  \section{programming}
    {\color{red} $\varheartsuit$} \textbf{j}ava\textbf{s}cript
    (ES5, node.js, Appcelerator Titanium)
    \textbf{p}er\textbf{l}, ba\textbf{sh}, \textbf{bat}ch C\#, \textbf{coffee}script
    \textbf{css}3, \textbf{html}5
	MySQL, MongoDB
\end{aside}

\section{interests}

ux, meetup, nosql, cutting edge technologies, everything JavaScript related stuff, build, deployment, nodejs, \textbf{beer}
%complex networks, social networks, community detection, community structure,
%overlapping communities, information diffusion, viral marketing, social
%inference, recommendation, data mining

\section{education}

\begin{entrylist}
  \entry
    {2011 - }
    {M.Sc.}
    {Corvinus University of Budapest}
	{Majoring in Business Information Technology}
  \entry
    {2007 - 2011}
    {B.Sc.}
    {Corvinus University of Budapest}
	{Majoring in Business Information Technology\\
	Specialization in e-Business}
  \entry
    {2001 - 2007}
    {Baccalaureate}
    {Katona József Gimnázium, Kecskemét}
    {Specialization in mathematics and IT}
		
\end{entrylist}

\section{experiences}

\begin{entrylist}
	\entry
		{Sep. 2011 - }
		{\href{http://creapps.net}{creApps Laboratory}}
		{owner, web and mobile developer}
		{\emph{In creApps we are working on our startup ideas, our current project is the \href{http://codrinking.net}{CoDrinking}.}}
	\entry
		{Sep. 2010 - }
		{\href{http://artanisdesign.eu/}{artanis.design}}
		{web and mobile developer}
		{\emph{Mostly project based development, including websites, mobile applications (iOS, Android, WP7) and complex systems (backend, frontend, mobile clients).}}
	\entry
		{Sep. 2009 -\\ Sep. 2011}
		{\href{http://camelcom.hu}{Cam-El-Com}}
		{web developer}
		{\emph{Development of an online dental practice management software, called \href{http://webflow-dental.com/}{webflow.dental}.}}
\end{entrylist}

\section{specialities}
First of all I love JavaScript.\\
Secondly I love experimenting.\\
\\
In 2009 (at Cam-El-Com) I started as a junior web backend developer. Mostly we were doing web development, but our complex system required much more beyond the web. So I did bash, batch for automation tasks, twain client in c\# for taking x-rays from Ireland directly to Hungary. The web part of the work was done with Perl (CGI, later Catalyst and at the end we were experimenting with Mojolicious), Apache, MySQL, memcached, HTML, jQuery.\\
\\
At artanis.design we have continued using Perl (Catalyst MVC, mod\_perl, mod\_psgi, reverse proxy based deployment with Apache).\\
\\
Beside backend development, I have started playing with mobile applications. For iOS and Android I'm using Appcelerator Titanium - a cross platform mobile solution written in JavaScript -, and for Windows Phone the native C\# and XAML.\\
\\
In the last two years I'm focusing mostly on frontend development (heavily JavaScript powered applications, also known as single page applications). I'm using jQuery, BackboneJS and require.js.\\
\\
Since October 2011 I'm experimenting with node.js and CoffeeScript with MongoDB. Currently we are using this architecture in an pre-production system.\\
\\
Now I'm getting familiar with build, deployment systems and PaaS.\\

\section{technologies}

\begin{entrylist}
  \entry
    {git}
    {version control system}
    {}
    {}
  \entry
    {PaaS}
    {platform-as-a-service}
    {}
    {I'm using \href{http://dotcloud.com}{dotCloud} since November 2011, and currently I'm getting know Amazon S3 and Heroku}
  \entry
	{UNIX}
	{Linux and OSX}
	{}
	{I prefer developing on linux and osx powered machines}
\end{entrylist}

%\section{publications}

%\printbibsection{article}{article in peer-reviewed journal}
%\printbibsection{inproceedings}{peer-revieed conference/proceedings}
%\printbibsection{misc}{other publications}
%\printbibsection{report}{research report}

\end{document}
